% Italian

\renewcommand\xCalendarTitle{Calendario Forest Sangha}

\renewcommand\xFirstDayOfVassa{Primo giorno di Vassa}
\renewcommand\xLastDayOfVassa{Ultimo giorno di Vassa}

\renewcommand\xAjahnChahMemorialDay{Commemorazione di Ajahn Chah}
\renewcommand\xAjahnChahBirthDay{Compleanno di Ajahn Chah}
\renewcommand\xThaiNewYear{Nuovo anno thailandese, Songkran}

\renewcommand\xBranchMonasteries{MONASTERI AFFILIATI}
\renewcommand\xWesternDisciplesOfAjahnChah{dei discepoli occidentali di Ajahn Chah}
\renewcommand\xPortalPageWorldWide{Il portale della comunità internazionale è:}

\renewcommand\xUnitedKingdom{Inghilterra}
\renewcommand\xSwitzerland{Svizzera}
\renewcommand\xThailand{Thailandia}
\renewcommand\xAustralia{Australia}
\renewcommand\xNewZealand{Nuova Zelanda}
\renewcommand\xUnitedStates{Stati Uniti}
\renewcommand\xItaly{Italia}
\renewcommand\xCanada{Canada}
\renewcommand\xPortugal{Portogallo}

\renewcommand\xDays{giorni}

% === Quotes ===

\SetTxt{January Quote}{%
La più nobile delle vie è l’ottuplice sentiero,\\
il più nobile discorso quello delle quattro nobili verità,\\
la libertà dal desiderio è il più nobile degli stati\\
e il Buddha che tutto vede il più nobile degli esseri.

\quoteRef{Dhammapada 273}
}

\SetTxt{February Quote}{%
Come un’ape raccogliendo il nettare\\
non nuoce né disturba il colore e il profumo del fiore\\
così il saggio si muove nel mondo.

\quoteRef{Dhammapada 49}
}

\SetTxt{March Quote}{%
Un solo verso autentico che calma la mente\\
è meglio di mille inconsistenti poesie.

\quoteRef{Dhammapada 101}
}

\SetTxt{April Quote}{%
Non atteggiarti a falsa umiltà. Segui con fermezza la tua meta.\\
La pratica diligente porta all'appagamento\\
sia nel presente che nel futuro.

\quoteRef{Dhammapada 168}
}

\SetTxt{May Quote}{%
Non cercare la compagnia di chi è sviato,\\
guardati da chi si è guastato.\\
Cerca la compagnia di amici fidati sulla Via,\\
di chi la visione profonda difende.

\quoteRef{Dhammapada 78}
}

\SetTxt{June Quote}{%
Contemplare la vita porta alla saggezza,\\
senza contemplazione la saggezza svanisce.\\
Discerni come la saggezza si coltiva e si distrugge\\
e cammina sulla via della crescita.

\quoteRef{Dhammapada 282}
}

\SetTxt{July Quote}{%
Chi ha disinnescato i rozzi attaccamenti\\
e gli attaccamenti sottili chi coltiva le facoltà spirituali\\
scopre la libertà dalla confusione.

\quoteRef{Dhammapada 370}
}

\SetTxt{August Quote}{%
Non far del male agli esseri viventi\\
che come noi cercano appagamento\\
significa far felici noi stessi

\quoteRef{Dhammapada 132}
}

\SetTxt{September Quote}{%
La vittoria porta all’odio\\
perché gli sconfitti soffrono.\\
Chi è in pace vive lieto\\
al di là di vittoria e sconfitta.

\quoteRef{Dhammapada 201}
}

\SetTxt{October Quote}{%
Una mente sana è il migliore guadagno.\\
L’appagamento è la risorsa più preziosa.\\
Un amico fidato è il migliore congiunto.\\
Una libertà senza condizioni è la massima beatitudine.

\quoteRef{Dhammapada 204}
}

\SetTxt{November Quote}{%
L’opportuna compagnia di amici è purezza.\\
Avere poche esigenze è purezza.\\
La virtù accumulata alla fine di una vita è purezza.\\
Essere al di là della sofferenza è purezza.

\quoteRef{Dhammapada 331}
}

\SetTxt{December Quote}{%
Tutto ciò che siamo è generato dalla mente. È la mente che traccia la strada.\\
Come la nostra ombra incessante ci segue, così ci segue il benessere\\
quando parliamo o agiamo con purezza di mente.

\quoteRef{Dhammapada 2}
}

% === Thumbnail Captions ===

\SetTxt{January Caption}{Monastero della Foresta Sītavana, Birken, Canada}

\SetTxt{February Caption}{Questua del mattino, Thailandia}

\SetTxt{March Caption}{Luang Por Anek istruisce\\ monaci in visita a\\ Wat Pah Sai Ngan,\\ NE Thailandia}

\SetTxt{April Caption}{Raduno Internazionale degli Anziani, 2017, Tempio di Amaravati, UK}

\SetTxt{May Caption}{Ajahn Karuṇadhammo (sinistra) e Ajahn Ñāṇiko (destra), Abhayagiri, USA}

\SetTxt{June Caption}{Ajahn Kevalī durante la questua, Bayreuth, vicino a Waldkloster Muttodaya, Germania}

\SetTxt{July Caption}{Abhayagiri, USA}

\SetTxt{August Caption}{Una \textit{sīladhara} (monaca) al monastero Amaravati}

\SetTxt{September Caption}{Luang Por Jundee in visita a un monastero cristiano,\\ Tomar, Portogallo}

\SetTxt{October Caption}{Cerimonia di Offerta della Veste, 2017, Aruna Ratanagiri, UK}

\SetTxt{November Caption}{Cucitura della veste a Cittaviveka, West Sussex, UK}

\SetTxt{December Caption}{Tempio di Amaravati, Hertfordshire, UK}

% === Frontmatter ===

\SetTxt{Frontmatter Text}{%
{\centering
\FrontmatterFmt
\mbox{}
\vfill

Questo calendario per il \CalendarYear\ è stato finanziato ai fini della distribuzione gratuita\\
dal gruppo Kataññutā di Malesia e Singapore.\\
Le foto sono state realizzate da una pluralità di fotografi.\\
Siamo a loro grati per il contributo offerto.

Foto di copertina: statua del Buddha in arenaria nell'altare del Monastero Abhayagiri, USA

Le citazioni presenti in ciascuna pagina sono interpretazioni in italiano di testi dal Canone Pāli.\\
Le traduzioni sono estratti della versione del Dhammapada: \emph{Dhammapada per la contemplazione},\\
\copyright\ Aruno Publications, scaricabile gratuitamente da:

https://forestsangha.org/teachings/books/dhammapada-per-la-contemplazione?language=Italiano

Vi preghiamo di tener presente che non sono traduzioni letterali.\\
Vi rimandiamo ad altre versioni per interpretazioni alternative.

Un ringraziamento va a tutti coloro che hanno contribuito alla realizzazione di questa pubblicazione.

\bigskip

{\large GIORNI LUNARI DI OSSERVANZA\hspace{5pt} \NewMoon\ \FirstQuarter\ \FullMoon\ \LastQuarter}

Nei monasteri, questi giorni sono dedicati a una maggiore introspezione. Persone in visita possono\\
partecipare a tutta o parte della prolungata meditazione serale e prendere i precetti.

Le date del calendario lunare sono ricavate in base a metodi tradizionali\\
e non sempre coincidono con quelle calcolate con i più moderni metodi astronomici.

\bigskip

{\large PRINCIPALI GIORNI DI LUNA PIENA PER IL \CalendarYear\ / \CalendarAltYear}

\emph{Māgha Pūjā} \spacedcdot\ \xDateMagha\ (“Giornata del Sangha”)\\
Commemora la riunione spontanea di 1250 arahant ai quali il Buddha\\
dette l'esortazione relativa agli aspetti basilari della Disciplina (\emph{Ovāda Pāṭimokkha}).

\emph{Vesākha Pūjā} \spacedcdot\ \xDateVesakha\ (“Giornata del Buddha”)\\
Commemora la nascita, l'illuminazione e la morte del Buddha.

\emph{Āsāḷhā Pūjā} \spacedcdot\ \xDateAsalha\ (“Giornata del Dhamma”)\\
Commemora il primo discorso del Buddha tenuto ai cinque \emph{samaṇa} nel Parco dei Cervi a Sarnath,\\
vicino a Varanasi. Il tradizionale Ritiro delle Stagione delle Piogge (\emph{Vassa}) inizia il giorno successivo.

\emph{Giorno di Pavāraṇā} \spacedcdot\ \xDatePavarana\\
Questo segna la fine dei tre mesi di ritiro del \emph{Vassa}. Durante il mese seguente,\\
la comunità laica potrà offrire la veste di \emph{Kaṭhina} come parte di una più\\
generale cerimonia di donazione.

\bigskip

{\large SITO WEB DELLA COMUNITÀ DEL NOSTRO SANGHA DELLA FORESTA}

www.forestsangha.org

\bigskip

Calendario ideato \& realizzato da Aruno Publications\\
Aruna Ratanagiri Buddhist Monastery\\
www.ratanagiri.org.uk

\copyright\ Aruno Publications \CopyrightYear\\

\vfill
\mbox{}
}}

