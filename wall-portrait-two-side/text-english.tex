% English

\renewcommand\xCalendarTitle{Forest Sangha Calendar}

\renewcommand\xFirstDayOfVassa{\mbox{First Day of Vassa}}
\renewcommand\xLastDayOfVassa{\mbox{Last Day of Vassa}}

\renewcommand\xAjahnChahMemorialDay{Ajahn Chah Memorial Day}
\renewcommand\xAjahnChahBirthDay{Ajahn Chah's Birthday}
\renewcommand\xThaiNewYear{Thai New Year, Songkran}

\renewcommand\xBranchMonasteries{BRANCH MONASTERIES}
\renewcommand\xWesternDisciplesOfAjahnChah{Western disciples of Ajahn Chah}
\renewcommand\xPortalPageWorldWide{The portal page for this community worldwide is:}

\renewcommand\xUnitedKingdom{United~Kingdom}
\renewcommand\xSwitzerland{Switzerland}
\renewcommand\xThailand{Thailand}
\renewcommand\xAustralia{Australia}
\renewcommand\xNewZealand{New~Zealand}
\renewcommand\xUnitedStates{United~States}
\renewcommand\xItaly{Italy}
\renewcommand\xCanada{Canada}
\renewcommand\xPortugal{Portugal}

\renewcommand\xDays{days}

% === Quotes ===

\SetTxt{January Quote}{%
\raggedright\color{white}%
Only blessings can arise from seeking\\
the company of wise and discerning persons,\\
who skilfully offer both admonition and advice\\
as if guiding one to hidden treasure.

\quoteRef{Dhammapada 76}
}

\SetTxt{February Quote}{%
\raggedright%
Blessed is the arising of a Buddha; blessed is the revealing of the Dhamma;\\
blessed is the concord of the Sangha; delightful is harmonious communion.

\quoteRef{Dhammapada 194}
}

\SetTxt{March Quote}{%
\raggedright%
Let the wise guide beings away from darkness,\\
give direction and advice.\\
They will be treasured by the virtuous\\
and dismissed by the foolish.

\quoteRef{Dhammapada 77}
}

\SetTxt{April Quote}{%
Tasting the flavour of solitude and the nectar of unshakeable peace,\\
those who drink the joy that is the essence of Truth abide free from fear of evil.

\quoteRef{Dhammapada 205}
}

\SetTxt{May Quote}{%
Devotion and respect should be offered to those\\
who have shown us the Way.

\quoteRef{Dhammapada 392}
}

\SetTxt{June Quote}{%
The sun shines by day, the moon shines by night.\\
But both all day and all night the Buddha shines\\
in glorious splendour.

\quoteRef{Dhammapada 387}
}

\SetTxt{July Quote}{%
A renunciate who abides in loving-kindness,\\
with a heart full of devotion for the Buddha’s teaching,\\
will find peace, stillness and bliss.

\quoteRef{Dhammapada 368}
}

\SetTxt{August Quote}{%
The Buddha's perfection is complete; there is no more work to be done.\\
No measure is there for his wisdom; no limits are there to be found.\\
In what way could he be distracted from truth?

\quoteRef{Dhammapada 179}
}

\SetTxt{September Quote}{%
\raggedright%
Having performed a wholesome deed it is good to repeat it, again and again.\\
Be interested in the pleasure of wholesomeness.\\
The fruit of accumulated goodness is contentment.

\quoteRef{Dhammapada 118}
}

\SetTxt{October Quote}{%
A renunciate does not oppress anyone.\\
Patient endurance is the ultimate asceticism.\\
Profound liberation, say the Buddhas, is the supreme goal.

\quoteRef{Dhammapada 184}
}

\SetTxt{November Quote}{%
One who transforms old and heedless ways into fresh and wholesome acts\\
brings light into the world, like the moon freed from clouds.

\quoteRef{Dhammapada 173}
}

\SetTxt{December Quote}{%
It is wisdom that leads to letting go of a lesser happiness\\
in pursuit of a happiness which is greater.

\quoteRef{Dhammapada 290}
}

% === Thumbnail Captions ===

\SetTxt{January Caption}{Wat Pah Cittaviveka, U.K.}

\SetTxt{February Caption}{Wat Kuean, NE Thailand}

\SetTxt{March Caption}{Wat Amaravati, U.K.}

\SetTxt{April Caption}{National Park, Kanchanaburi, W.~Thailand}

\SetTxt{May Caption}{Wat Santacittarama, Italy}

\SetTxt{June Caption}{Vesākha Puja circumambulation ceremony, Janamara Hermitage, Thailand}

\SetTxt{July Caption}{Wat Amaravati, U.K.}

\SetTxt{August Caption}{Wat Aruna Ratanagiri, U.K.}

\SetTxt{September Caption}{Birken Monastery, Canada}

\SetTxt{October Caption}{Wat Pah Nanachat, NE~Thailand}

\SetTxt{November Caption}{Wat Amaravati, U.K.}

\SetTxt{December Caption}{Wat Aruna Ratanagiri, U.K.}

% === Frontmatter ===

\SetTxt{Frontmatter Text}{%
{\centering
\FrontmatterFmt
\mbox{}
\vfill

This \CalendarYear\ calendar has been sponsored for free distribution\\
by the Kataññutā group of Malaysia, Singapore and Australia.\\
It features pictures by a variety of photographers.\\
We are grateful for their generous contribution.

Cover: photo of the Buddha rupa in the Uposatha Hall at Wat Pah Nanachat, NE Thailand

Scriptural quotes on each page are English renderings of texts from the Pali Canon.\\
The translations are extracts from the Dhammapada interpretation:\\
\emph{A Dhammapada for Contemplation}, \copyright\ Aruno Publications,\\
available for free download at:

https://forestsangha.org/teachings/books/a-dhammapada-for-contemplation?language=English

Please note that these are not literal translations. For further renderings please refer to other works.

Appreciation is expressed to all who have offered assistance with this production.

\bigskip

{\large LUNAR OBSERVANCE DAYS\hspace{5pt} \NewMoon\ \FirstQuarter\ \FullMoon\ \LastQuarter}

These days are devoted to quiet reflection at the monastery. Visitors may come and take the\\
precepts for the day and join in all or part of the extended evening meditation.

The dates for the lunar calendar are determined by traditional methods of calculation,\\
and are therefore not always the same as the precise astronomical occurrences.

\bigskip

{\large THE MAJOR FULL MOON DAYS FOR \CalendarYear\ / \CalendarAltYear}

\emph{Māgha Pūjā} \spacedcdot\ \xDateMagha\ (`Sangha Day')\\
Commemorates the spontaneous gathering of 1250 arahants to whom the Buddha\\
gave the exhortation on the basis of the Discipline (\emph{Ovāda Pāṭimokkha}).

\emph{Vesākha Pūjā} \spacedcdot\ \xDateVesakha\ (`Buddha Day')\\
Commemorates the birth, enlightenment and passing away of the Buddha.

\emph{Āsāḷhā Pūjā} \spacedcdot\ \xDateAsalha\ (`Dhamma Day')\\
Commemorates the Buddha's first discourse, given to the five \emph{samaṇas}\\
in the Deer Park at Sarnath, near Varanasi. The traditional\\
Rainy-Season Retreat (\emph{Vassa}) begins on the next day.

\emph{Pavāraṇā Day} \spacedcdot\ \xDatePavarana\\
This marks the end of the three-month \emph{Vassa} retreat. In the following month,\\
lay people may offer the \emph{Kaṭhina}-robe as part of a general alms-giving ceremony.

\bigskip

{\large WEB ADDRESS FOR THIS FOREST SANGHA COMMUNITY}

www.forestsangha.org

\bigskip

Calendar design \& production by Aruno Publications\\
Aruna Ratanagiri Buddhist Monastery\\
www.ratanagiri.org.uk

\copyright\ Aruno Publications \CopyrightYear\\

\vfill
\mbox{}
}}

