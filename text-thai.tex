% Thai

\renewcommand\xCalendarTitle{ปฏิทินคณะสงฆ์สายวัดป่า}

\renewcommand\xHemanta{ฤดูหนาว}
\renewcommand\xGimha{ฤดูร้อน}
\renewcommand\xVassana{ฤดูฝน}

\renewcommand\xMaghaPuja{วันมาฆบูชา}
\renewcommand\xVesakhaPuja{วันวิสาขบูชา}
\renewcommand\xAsalhaPuja{วันอาสาฬหบูชา}
\renewcommand\xPavarana{วันปวารณา}

\renewcommand\xFirstDayOfVassa{\mbox{วันแรกของการเข้าพรรษา}}
\renewcommand\xLastDayOfVassa{\mbox{วันสุดท้ายของการเข้าพรรษา}}

\renewcommand\xAjahnChahMemorialDay{วันครบรอบการมรณภาพของหลวงพ่อชา}
\renewcommand\xAjahnChahBirthDay{วันคล้ายวันเกิดของหลวงพ่อชา}
\renewcommand\xThaiNewYear{วันสงกรานต์}

\renewcommand\xBranchMonasteries{วัดสาขา}
\renewcommand\xWesternDisciplesOfAjahnChah{ลูกศิษย์ชาวตะวันตกของหลวงพ่อชา}
\renewcommand\xPortalPageWorldWide{หน้า Portal สำหรับคณะสงฆ์ทั่วโลกคือ:}

\renewcommand\xUnitedKingdom{สหราชอาณาจักร}
\renewcommand\xSwitzerland{สวิตเซอร์แลนด์}
\renewcommand\xThailand{ประเทศไทย}
\renewcommand\xAustralia{ออสเตรเลีย}
\renewcommand\xNewZealand{นิวซีแลนด์}
\renewcommand\xUnitedStates{สหรัฐอเมริกา}
\renewcommand\xItaly{อิตาลี}
\renewcommand\xCanada{แคนาดา}
\renewcommand\xPortugal{โปรตุเกส}

\renewcommand\xDays{ค่ำ}

% === Quotes ===

\SetTxt{January Quote}{%
มนุษย์ทั้งหลายเกิดมามีอิสระ และ เสมอภาคกันในเกียรติศักด และ สิทธิ\\
ต่างมีเหตุผล และ มโนธรรม และ ควรปฏิบัติต่อกันด้วยเจตนารมณ์แห่ง\\
ภราดรภาพ
}

\SetTxt{February Quote}{%
มนุษย์ทั้งหลายเกิดมามีอิสระ และ เสมอภาคกันในเกียรติศักด และ สิทธิ\\
ต่างมีเหตุผล และ มโนธรรม และ ควรปฏิบัติต่อกันด้วยเจตนารมณ์แห่ง\\
ภราดรภาพ
}

\SetTxt{March Quote}{%
มนุษย์ทั้งหลายเกิดมามีอิสระ และ เสมอภาคกันในเกียรติศักด และ สิทธิ\\
ต่างมีเหตุผล และ มโนธรรม และ ควรปฏิบัติต่อกันด้วยเจตนารมณ์แห่ง\\
ภราดรภาพ
}

\SetTxt{April Quote}{%
มนุษย์ทั้งหลายเกิดมามีอิสระ และ เสมอภาคกันในเกียรติศักด และ สิทธิ\\
ต่างมีเหตุผล และ มโนธรรม และ ควรปฏิบัติต่อกันด้วยเจตนารมณ์แห่ง\\
ภราดรภาพ
}

\SetTxt{May Quote}{%
มนุษย์ทั้งหลายเกิดมามีอิสระ และ เสมอภาคกันในเกียรติศักด และ สิทธิ\\
ต่างมีเหตุผล และ มโนธรรม และ ควรปฏิบัติต่อกันด้วยเจตนารมณ์แห่ง\\
ภราดรภาพ
}

\SetTxt{June Quote}{%
มนุษย์ทั้งหลายเกิดมามีอิสระ และ เสมอภาคกันในเกียรติศักด และ สิทธิ\\
ต่างมีเหตุผล และ มโนธรรม และ ควรปฏิบัติต่อกันด้วยเจตนารมณ์แห่ง\\
ภราดรภาพ
}

\SetTxt{July Quote}{%
มนุษย์ทั้งหลายเกิดมามีอิสระ และ เสมอภาคกันในเกียรติศักด และ สิทธิ\\
ต่างมีเหตุผล และ มโนธรรม และ ควรปฏิบัติต่อกันด้วยเจตนารมณ์แห่ง\\
ภราดรภาพ
}

\SetTxt{August Quote}{%
มนุษย์ทั้งหลายเกิดมามีอิสระ และ เสมอภาคกันในเกียรติศักด และ สิทธิ\\
ต่างมีเหตุผล และ มโนธรรม และ ควรปฏิบัติต่อกันด้วยเจตนารมณ์แห่ง\\
ภราดรภาพ
}

\SetTxt{September Quote}{%
มนุษย์ทั้งหลายเกิดมามีอิสระ และ เสมอภาคกันในเกียรติศักด และ สิทธิ\\
ต่างมีเหตุผล และ มโนธรรม และ ควรปฏิบัติต่อกันด้วยเจตนารมณ์แห่ง\\
ภราดรภาพ
}

\SetTxt{October Quote}{%
มนุษย์ทั้งหลายเกิดมามีอิสระ และ เสมอภาคกันในเกียรติศักด และ สิทธิ\\
ต่างมีเหตุผล และ มโนธรรม และ ควรปฏิบัติต่อกันด้วยเจตนารมณ์แห่ง\\
ภราดรภาพ
}

\SetTxt{November Quote}{%
มนุษย์ทั้งหลายเกิดมามีอิสระ และ เสมอภาคกันในเกียรติศักด และ สิทธิ\\
ต่างมีเหตุผล และ มโนธรรม และ ควรปฏิบัติต่อกันด้วยเจตนารมณ์แห่ง\\
ภราดรภาพ
}

\SetTxt{December Quote}{%
มนุษย์ทั้งหลายเกิดมามีอิสระ และ เสมอภาคกันในเกียรติศักด และ สิทธิ\\
ต่างมีเหตุผล และ มโนธรรม และ ควรปฏิบัติต่อกันด้วยเจตนารมณ์แห่ง\\
ภราดรภาพ
}

% === Thumbnail Captions ===

\SetTxt{January Caption}{มนุษย์ทั้งหลายเกิดมามีอิสระ และ เสมอภาคกันในเกียรติศักด และ สิทธิ}

\SetTxt{February Caption}{มนุษย์ทั้งหลายเกิดมามีอิสระ และ เสมอภาคกันในเกียรติศักด และ สิทธิ}

\SetTxt{March Caption}{มนุษย์ทั้งหลายเกิดมามีอิสระ และ เสมอภาคกันในเกียรติศักด และ สิทธิ}

\SetTxt{April Caption}{มนุษย์ทั้งหลายเกิดมามีอิสระ และ เสมอภาคกันในเกียรติศักด และ สิทธิ}

\SetTxt{May Caption}{มนุษย์ทั้งหลายเกิดมามีอิสระ และ เสมอภาคกันในเกียรติศักด และ สิทธิ}

\SetTxt{June Caption}{มนุษย์ทั้งหลายเกิดมามีอิสระ และ เสมอภาคกันในเกียรติศักด และ สิทธิ}

\SetTxt{July Caption}{มนุษย์ทั้งหลายเกิดมามีอิสระ และ เสมอภาคกันในเกียรติศักด และ สิทธิ}

\SetTxt{August Caption}{มนุษย์ทั้งหลายเกิดมามีอิสระ และ เสมอภาคกันในเกียรติศักด และ สิทธิ}

\SetTxt{September Caption}{มนุษย์ทั้งหลายเกิดมามีอิสระ และ เสมอภาคกันในเกียรติศักด และ สิทธิ}

\SetTxt{October Caption}{มนุษย์ทั้งหลายเกิดมามีอิสระ และ เสมอภาคกันในเกียรติศักด และ สิทธิ}

\SetTxt{November Caption}{มนุษย์ทั้งหลายเกิดมามีอิสระ และ เสมอภาคกันในเกียรติศักด และ สิทธิ}

\SetTxt{December Caption}{มนุษย์ทั้งหลายเกิดมามีอิสระ และ เสมอภาคกันในเกียรติศักด และ สิทธิ}

% === Frontmatter ===

\SetTxt{Frontmatter Text}{%
{\centering
\FrontmatterFmt
\THSarabunNewFont\selectfont
\mbox{}
\vfill

ปฏิทินปี \CalendarYear\ นี้มีภาพถ่ายจากช่างภาพหลากหลายท่าน\\
เราขอขอบคุณสำหรับการสนับสนุนที่กรุณาของพวกเขา

\vspace*{2\baselineskip}

{\large วันพระวันอุโบสถ\hspace{5pt} \NewMoon\ \FirstQuarter\ \FullMoon\ \LastQuarter}

วันเหล่านี้จัดขึ้นเพื่อการสงบจิตสงบใจและพิจารณาธรรมที่วัด
ท่านผู้มาวัดสามารถมารับศีลและร่วมนั่งสมาธิปฏิบัติธรรมทั้งวัน
จะเข้าร่วมบางเวลาหรือตลอดทั้งวันจนถึงทำวัตรสวดมนต์เย็นก็ได้

วันที่ในปฏิทินจันทรคติได้ใช้การคำนวณตามวิธีแบบดั้งเดิม
จึงอาจไม่ตรงกับปรากฏการณ์ทางดาราศาสตร์อย่างเที่ยงตรงเสมอไป

สำหรับข้อมูลเพิ่มเติม สามารถเยี่ยมชมได้ที่: https://splendidmoons.github.io

\bigskip

{\large วันพระ วันอุโบสถที่สำคัญในปี \CalendarYear\ / \CalendarAltYear}

มาฆบูชา \spacedcdot\ \xDateMagha\ (`วันแห่งพระสงฆ์')\\
ระลึกถึงการประชุมของพระอรหันต์ 1,250 รูปอย่างพร้อมเพรียงกันโดยมิได้นัดหมาย\\
ซึ่งพระพุทธเจ้าได้ทรงแสดงโอวาทปาฏิโมกข์อันเป็นหลักแห่งพระวินัยสังฆ์

วิสาขบูชา \spacedcdot\ \xDateVesakha\ (`วันแห่งพระพุทธเจ้า')\\
ระลึกถึงการประสูติ ตรัสรู้ และปรินิพพานของพระพุทธเจ้า

อาสาฬหบูชา \spacedcdot\ \xDateAsalha\ (`วันแห่งพระธรรม')\\
ระลึกถึงพระธรรมเทศนาครั้งแรกของพระพุทธเจ้าที่แสดงแก่ปัญจวัคคีย์ในป่าอิสิปตนมฤคทายวัน\\
ใกล้เมืองพาราณสี วันเข้าพรรษาเริ่มในวันถัดไป

วันปวารณา \spacedcdot\ \xDatePavarana\\
เป็นวันสิ้นสุดการจำพรรษาสามเดือน ในเดือนถัดไป\\
พุทธศาสนิกชนสามารถถวายผ้ากฐินซึ่งเป็นส่วนหนึ่งของพิธีถวายทานทั่วไปได้

\vspace*{2\baselineskip}

{\large เว็บไซด์ของคณะสงฆ์สายวัดป่า}

www.forestsangha.org

\vfill
\mbox{}
\normalfont\selectfont
}}

