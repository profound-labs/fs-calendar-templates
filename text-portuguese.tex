% Portuguese

\renewcommand\xCalendarTitle{Calendário do Sangha da Floresta}

\renewcommand\xFirstDayOfVassa{\mbox{Primeiro dia do Vassa}}
\renewcommand\xLastDayOfVassa{\mbox{Último dia do Vassa}}

\renewcommand\xAjahnChahMemorialDay{Dia em memória de Ajahn Chah}
\renewcommand\xAjahnChahBirthDay{Aniversário de Ajahn Chah}
\renewcommand\xThaiNewYear{Ano Novo Tailandês, Songkran}

\renewcommand\xBranchMonasteries{MOSTEIROS ASSOCIADOS}
\renewcommand\xWesternDisciplesOfAjahnChah{Discípulos ocidentais de Ajahn Chah}
\renewcommand\xPortalPageWorldWide{Página portal da Intenet para esta comunidade a nível mundial:}

\renewcommand\xUnitedKingdom{Reino~Unido}
\renewcommand\xSwitzerland{Suiça}
\renewcommand\xThailand{Tailândia}
\renewcommand\xAustralia{Austrália}
\renewcommand\xNewZealand{Nova~Zelândia}
\renewcommand\xUnitedStates{Estados~Unidos}
\renewcommand\xItaly{Itália}
\renewcommand\xCanada{Canadá}
\renewcommand\xPortugal{Portugal}

\renewcommand\xDays{dias}

% === Quotes ===

\SetTxt{January Quote}{%
O Caminho Óctuplo é o caminho mais honrável,\\
as Quatro Nobres Verdades o mais honrável enunciado,\\
estar liberto do desejo o mais honrável estado,\\
e o Buddha, que tudo vê, o mais honrado Ser.

\quoteRef{Dhammapada 273}
}

\SetTxt{February Quote}{%
Tal como uma abelha recolhendo o néctar\\
não danifica ou perturba a cor e a fragância da flor,\\
também assim os sábios se movem pelo mundo.

\quoteRef{Dhammapada 49}
}

\SetTxt{March Quote}{%
É melhor ouvir uma só palavra verdadeira\\
que acalme a mente, do que milhares de palavras irrelevantes.\\

\quoteRef{Dhammapada 101}
}

\SetTxt{April Quote}{%
Não mostres falsa humildade. Mantém-te firme em relação ao teu objectivo.\\
A práctica bem levada a cabo, conduz ao contentamento,\\
agora e no futuro.

\quoteRef{Dhammapada 168}
}

\SetTxt{May Quote}{%
Não procures a companhia de amigos desorientados;\\
tem cautela com companhias degeneradas.\\
Busca a companhia de companheiros bem orientados,\\
aqueles que oferecem apoio à realização.

\quoteRef{Dhammapada 78}
}

\SetTxt{June Quote}{%
Contemplar a vida traz Sabedoria; sem contemplação a sabedoria mirra.\\
Reconhece de que forma a sabedoria é cultivada, de que forma é destruída,\\
e percorre o caminho do crescimento.

\quoteRef{Dhammapada 282}
}

\SetTxt{July Quote}{%
Aquele que acabou com os apegos mais grosseiros,\\
que acabou com os apegos mais subtis, que cultiva as qualidades espirituais,\\
é aquele que se liberta da ilusão.

\quoteRef{Dhammapada 370}
}

\SetTxt{August Quote}{%
Evitar prejudicar seres vivos que,\\
tal como nós, procuram contentamento,\\
é trazer felicidade para nós próprios. 

\quoteRef{Dhammapada 132}
}

\SetTxt{September Quote}{%
A vitória leva ao ódio do sofredor vencido.\\
O pacífico vive feliz\\
para além da vitória e da derrota.

\quoteRef{Dhammapada 201}
}

\SetTxt{October Quote}{%
Uma mente saudável é o maior ganho.\\
Contentamento é a maior riqueza.\\
Confiabilidade é o melhor parente.\\
Liberdade incondicional é a mais elevada bênção.

\quoteRef{Dhammapada 204}
}

\SetTxt{November Quote}{%
A oportuna companhia de amigos é uma bênção.\\
Escassez de necessidades é uma bênção.\\
Ter virtude acumulada no fim da vida é uma bênção.\\
Ter-se libertado de todo o sofrimento é uma bênção.

\quoteRef{Dhammapada 331}
}

\SetTxt{December Quote}{%
Todos os estados de ser são determinados pelo coração. É o coração que lidera o caminho.\\
Tão certo como a nossa sombra nunca nos deixar, também o bem-estar surgirá\\
quando falamos ou agimos de coração puro.

\quoteRef{Dhammapada 2}
}

% === Thumbnail Captions ===

\SetTxt{January Caption}{Mosteiro da Floresta Sītavana, Birken, Canadá}

\SetTxt{February Caption}{Ronda da mendicância matinal, Tailândia}

\SetTxt{March Caption}{Luang Por Anek instruíndo\\ monges de visita a\\ Wat Pah Sai Ngan,\\ Nordeste da Tailândia}

\SetTxt{April Caption}{Reunião Internacional dos Anciões, 2017, Templo de Amaravati, Reino Unido}

\SetTxt{May Caption}{Ajahn Karuṇadhammo (esquerda) e Ajahn Ñāṇiko (direita), Abhayagiri, EUA}

\SetTxt{June Caption}{Ajahn Kevalī na ronda da mendicância, Bayreuth, perto de Waldkloster Muttodaya, Alemanha}

\SetTxt{July Caption}{Abhayagiri, EUA}

\SetTxt{August Caption}{Uma \textit{sīladhara} (monja) no Mosteiro de Amaravati}

\SetTxt{September Caption}{Luang Por Jundee de visita ao Convento de Cristo,\\ Tomar, Portugal}

\SetTxt{October Caption}{Cerimónia da oferta dos mantos, 2017, Aruna Ratanagiri, Reino Unido}

\SetTxt{November Caption}{Sessão de costura em Cittaviveka, West Sussex, Reino Unido}

\SetTxt{December Caption}{Templo de Amaravati, Hertfordshire, Reino Unido}

% === Frontmatter ===

\SetTxt{Frontmatter Text}{%
{\centering
\FrontmatterFmt
\mbox{}
\vfill

Este calendário de \CalendarYear\ foi patrocinado para distribuição gratuita\\
pelo grupo Kataññutā da Malásia e Singapura.\\
Apresenta fotografias de vários fotógrafos, aos quais agradecemos\\
pela suas generosas contribuições.

Capa: fotografia da imagem do Buddha em arenito, no altar do Mosteiro Budista Abhayagiri, EUA

As citações das escrituras apresentadas são traduções em Português de textos do Cânone Pali.
Estas são traduções extraídas da interpretação do Dhammapada presente em 
\emph{A Dhammapada for Contemplation}, \copyright\ Aruno Publications, com tradução
em Português disponível para download gratuito em:

http://sumedharama.pt/biblioteca/livros-2/livros/

Estas não são traduções literais e outras traduções podem ser encontradas em obras distintas.

A nossa gratidão a todos os que contribuíram para a realização desta edição.

\bigskip

{\large DIAS DE CERIMÓNIA LUNAR\hspace{5pt} \NewMoon\ \FirstQuarter\ \FullMoon\ \LastQuarter}

Estes dias são regularmente dedicados a reflexão silenciosa no mosteiro.\\
As datas para o calendário lunar são determinadas por métodos de cálculo tradicional\\
e nem sempre são no mesmo dia das ocorrências astronómicas.

\bigskip

{\large OS DIAS MAIS IMPORTANTES DE LUA CHEIA PARA \CalendarYear\ / \CalendarAltYear}

\emph{Māgha Pūjā} \spacedcdot\ \xDateMagha\ (`Dia do Sangha')\\
Comemora a reunião espontânea de 1250 arahants a quem o Buda ofereceu\\
uma exortação com base na Disciplina (\emph{Ovāda Pāṭimokkha}).

\emph{Vesākha Pūjā} \spacedcdot\ \xDateVesakha\ (`Dia do Buda')\\
Comemora o nascimento, iluminação e morte do Buda.

\emph{Āsāḷhā Pūjā} \spacedcdot\ \xDateAsalha\ (`Dia do Dhamma')\\
Comemora o primeiro discurso de Buda, dado aos cinco \emph{samaṇas} no Parque dos Veados em Sarnat,\\
perto de Varanasi. O tradicional Retiro da Estação das Chuvas (\emph{Vassa}) começa no dia seguinte.

\emph{Pavāraṇā Day} \spacedcdot\ \xDatePavarana\\
Este dia marca o fim do retiro de três meses do \emph{Vassa}. Durante o mês seguinte,\\
a comunidade alargada de apoiantes dos mosteiros, oferece tradicionalmente o pano para\\
o hábito monástico, como parte da Cerimónia de Doações da época da \emph{Kaṭhina}.

\bigskip

{\large SÍTIO DA INTERNET PARA ESTA COMUNIDADE DO SANGHA DA FLORESTA}

www.forestsangha.org

\bigskip

Produção e design do calendário por Aruno Publications\\
Aruna Ratanagiri Buddhist Monastery, Reino Unido\\
www.ratanagiri.org.uk

\copyright\ Aruno Publications \CopyrightYear\\

\vfill
\mbox{}
}}

